% arara: pdflatex: { options: "--jobname great-name" }
\documentclass[12pt,a4paper]{report}
\usepackage[utf8]{inputenc}
\usepackage[margin=1.8cm]{geometry}
\usepackage{amsmath}
\usepackage[czech, slovak]{babel}
\usepackage{amsfonts}
\usepackage{amssymb}
\usepackage{graphicx}
\usepackage{lmodern}

% TODO
% -


\begin{document}



\begin{titlepage}

\fontfamily{qhv}\selectfont
\addtolength{\voffset}{-3cm}

\noindent\hspace{-60.0pt}\includegraphics[width=0.6\textwidth]{./img/VUT_symbol_barevne_CMYK_CZ}\\
\Large\textbf{VYSOKÉ UČENÍ TECHNICKÉ V BRNĚ}\\
\small\textbf{BRNO UNIVERSITY OF TECHNOLOGY}\\\vfill\hspace{-0.7cm}
\Large\textbf{FAKULTA INFORMAČNÍCH TECHNOLOGIÍ}\\
\small\textbf{FACULTY OF INFORMATION TECHNOLOGY}\\\vfill\hspace{-0.7cm}
\Large\textbf{IFJ - DOKUMENTACE PROJEKTU}\\
\small\textbf{IFJ - PROJECT DOCUMENTATION}\\\vfill\hspace{-0.7cm}
\Large\textbf{SEMESTRÁLNÍ PROJEKT}\\
\small\textbf{TERM PROJECT}\\\vfill\hspace{-0.7cm}
\Large\textbf{AUTORI PRÁCE \hfill \normalsize{Marek Tamaškovič, Martin Vaško,}}\\
\small\textbf{AUTHORS} \hfill \normalsize\textbf{Michal Vaško, Jirka Záleský, Jakub Zárybnický}
\vfill\hspace{-0.7cm}
\large\textbf{BRNO 2017}

\newpage
\fontfamily{\familydefault}\selectfont%
\end{titlepage}

\newgeometry{margin=2.5cm}

\begin{abstract}
Nas abstrakt\\
http://www.fit.vutbr.cz/info/szz/bib\_citace.html\\
http://www.fit.vutbr.cz/info/szz/psani\_textu.php\\
http://citace.info/norma1/webova-stranka/
\end{abstract}

\tableofcontents

\chapter{Implementačné detaily}

\section{Lexikálna analýza}
Lexikálna analýza alebo teda Scanner funguje na princípe konečného automatu. Postupne načitáva  znaky zo vstupného súboru a posiela ich syntaktickému analýzatoru vo forme tokenov.Poslany token obsahuje informácie o type, obsahu a pozicii(riadok,stĺpec) na ktorej sa v interpretovanom subore nachádza.Všetky tieto informácie su potrebné pre uľahčenie práce parseru.V prípade ,že načítame niečo čo nezapadá do pravidiel programovacieho jazyka IFJ16 nastane lexikálna chyba. Náš scanner podporuje aj všetky dostupne a povolené rozšírenia ako napríklad unárne operatory či rozširenie BASE a dalšie. Diagram konečnéh automatu si možete prezrieť v priloženej prílohe. Je robený čo najprehladnejšie a preto došlo k par upravám ktoré uplne nezodpovedajú stavom v kóde.
\section{Syntaktická analýza}
\textbf{\textit{TODO}}
\section{Sémantická analýza}
\textbf{\textit{TODO}}
\section{Interpret}
Interpret má za úlohu vykonať to, čo sa nachádza v zdrojovom kóde interpretovaného programu. Náš interpret interpretuje abstraktný syntaktický strom, ktorý vytvorila syntaktická analýza. Ten lineárne prechádza a vyhodnocuje výrazy pokým nenastane volanie funkcie. V tom momente si vyhľadá v tabuľke symbolov abstraktný syntaktický strom danej funkcie, vytvorí lokálnu tabuľku symbolov, ktorú bude používať volaná funkcia, vloží do nej argumenty funkcie a začne vykonávať telo funkcie. Pri ukončovaní funkcie interpret vloží návratovú hodnotu z funkcie na zásobník a ukončí interpretáciu funkcie. Následne si interpret danú hodnotu vyberie zo zásobníka a použije ju v interpretácii pôvodného abstraktného syntaktického stromu. Vstavané funkcie sú riešené obdobne. Taktiež si pri interpretácii interpret kontroluje behové chyby ako napríklad delenie nulou alebo práca s neinicializovanými premennými. Ak taká situácia nastane interpret sa ukončuje s chybovou hláškou a príslušným návratovým kódom pre danú chybu.


\section{Algoritmy}
% Nejak zvýraznit názvy funkcí?
\subsection{List-Merge sort}
\par Jako řadící algoritmus jsme využili algoritmu typu divide et impera (angl. Divide and Conquer) zvaný Merge sort. Princip algoritmu spočívá v postupném rozdělení pole na podpole, která následně po dvojicích spojujeme tak, aby následně pole vzniklé spojením bylo seřazené. Takto postupujeme do te doby dokud, není seřazeno celé pole. Časová složitost algoritmu je $\mathcal{O}(n\log{}n)$
\par Naše implementace využívá rekurzivního volání samotné funkce sort. Celý proces je rozdělen na dvě fáze. V první fázi každá rekurzivně volaná funkce využívá vestavěné funkce interpretu substr pro rozdělení pole na podpole a to dokud nejsou všechna podpole velikosti 1. Následně započne fáze skládání, kdy se podle váhy vybírají prvky z podpolí a řadí v odpovídajícím pořadí.

\subsection{KMP Vyhledání podřetězce}
\par Knuth-Morris-Prattův algoritmus se využívá pro urychlení hledání výskytu stejných podřetězců v hledaném řetězci. Algoritmus využívá pomocného pole obsahující informace o hledaném podřetězci. Toto pole musí být vytvořeno ještě předtím než je zahájeno hledání. Složitost algoritmu je $\mathcal{O}(n)$.
\par V první fázi je zavolána pomocná funkce Prefixcreator, která do pomocného pole uchovávajícího pozice od kterých se má pokračovat ve porovnávání, umístí hodnoty na základě nalezených podřetězců. Není-li podřetězec větší než 1 pak je pomocná tabulka vyplněna -1.
Ve fázi druhé již probíhá porovnávání hledaného řetězce s řetězcem vstupním. Výsledkem je pozice na, které začíná hledaný řetězec.

\chapter{Príloha}
\includegraphics[width=1\textwidth]{./img/lex1.png}\\


\end{document}
