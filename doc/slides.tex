\documentclass[13pt]{beamer}
\usetheme{CambridgeUS}
\usecolortheme{dolphin}

\usepackage{microtype}
\usepackage{amsmath}
\usepackage{lmodern}
\usepackage{tikz}
\usepackage[czech,slovak]{babel}
\usepackage[utf8x]{inputenc}
\usepackage{times}

\usetikzlibrary{calc,trees,positioning,arrows,chains,shapes.geometric,%
    decorations.pathreplacing,decorations.pathmorphing,shapes,%
    matrix,shapes.symbols}

\title{Interpret jazyka IFJ16}
\subtitle{Semestrální projekt IFJ}
\institute{FIT VUT Brno}
\author[ Vaško,Vaško,Zárybnický,Záleský,Tamaškovič]
{Michal~Vaško, Martin~Vaško, Jakub~Zárybnický, Jiří~Záleský, Marek~Tamaškovič}
\date{14. prosince 2016}

\begin{document}

\begin{frame}
  \titlepage
\end{frame}

\begin{frame}{Obsah}
  \tableofcontents
\end{frame}

\section{Úvod}

\subsection{Zadání}

\begin{frame}{Zadání}
\end{frame}

\section{Architektura}

\begin{frame}{Přehled architektury}
\begin{center}
\begin{tikzpicture}
[node distance = 1cm, auto,,
every node/.style={node distance=2cm},
force/.style={rectangle, rounded corners, draw=black, very thick,
text width=8em, text badly centered, minimum height=3em}]

% Draw forces
\node [force] (center) {Hlavní řadič};
\node [force, left=.5cm of center] (database) {Databáze};
\node [force, below=.5cm of center] (gui) {Grafické rozhraní};
\node [force, below=.5cm of gui] (user) {Uživatel};
\node [force, right=.5cm of center] (server) {Server};

\path[->,thick,shorten >=1pt,shorten <=1pt]
(center) edge (database)
(center) edge (gui)
(center) edge (server)
(gui) edge (center)
(server) edge (center)
(database) edge (center)
(user) edge (gui)
(gui) edge (user);

\end{tikzpicture}
\end{center}
\end{frame}

\begin{frame}{Datové typy (AST)}
\end{frame}

\begin{frame}{Hierarchie typů}
\end{frame}

\section{Komponenty}
\subsection{Lexer}

\begin{frame}{Poznámky k implementaci (změnit název?)}
počítání line, char, BASE
\end{frame}

\begin{frame}{Stavový diagram lexeru}
\begin{center}
\begin{tikzpicture}
  [>=stealth',
  punktchain/.style={
    rectangle,
    rounded corners,
    draw=black, very thick,
    text width=10em,
    minimum height=2.5em,
    text centered,
    on chain},
  every join/.style={->, thick,shorten >=1pt},
  node distance=.5cm, start chain=going below,]
  \node[punktchain, join] {Anal\'yza};
  \node[punktchain, join] {Návrh};
  \node[punktchain, join] {Implementace};
  \node[punktchain, join] {Testov\'an\'i};
  \node[punktchain, join] {Publikování};
\end{tikzpicture}
\end{center}
\end{frame}

\subsection{Parser}

\begin{frame}{Poznámky k implementaci (změnit název?)}
UNARY, re-re-re-implemeentace, makra
\end{frame}

\begin{frame}{LL gramatika}
\end{frame}

\begin{frame}{Precedenční tabulka}
\end{frame}

\subsection{Sémantická analýza}

\begin{frame}{Zaujimavosti v implementácii ???}
  \begin{itemize}
    \item AST
    \item V projekte používame \'funkcionálne\' programovanie.
    \item Nachádzajú sa tam rôzne kontroly napr.: \ttfamily getExpressionType, isAssignCompatible
    \sffamily
  \end{itemize}
funkcionální styl, příklady různých kontrol (break, getExpressionType, isAssignCompatible)?
\end{frame}

\subsection{Interpret}

\begin{frame}{Poznámky k implementaci (změnit název?)}

  \begin{itemize}
    \item \ttfamily \par\#define I(x) (x)->data.integer
    \item \sffamily Telo cyklu je makro (cykly sa líšia len v tom čo sa deje pred alebo po vykonaní tela cyklu)
    \item AntiDijsktra kód (Nachádza sa v ňom 3 krát goto, ktoré nevedie k erroru)
  \end{itemize}

%S(x), cykly, goto vs Dijkstra (reference: https://dl.acm.org/citation.cfm?id=362947), GC
\end{frame}

\section{Algoritmy}

\begin{frame}{Algoritmy...}
\end{frame}

\section{Závěr}

\begin{frame}{Statistiky}
\end{frame}

\begin{frame}{Očekávání, rozpočet}
31/10
\end{frame}

\begin{frame}{Zhodnocení}
jazykové bariéry, ?
\end{frame}

\end{document}
